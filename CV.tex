\documentclass{muratcan_cv}
\usepackage {ctex}
\usepackage[utf8]{inputenc}
\setname{\textbf{\fontsize{22pt}{24pt}张昕杰}}{}
\setaddress{湖北省武汉市}
\setmobile{18150153576}
\setmail{zjie5163@gmail.com}
\setposition{本科三年级} %ignored for now
%\setbloguarl{http://raymoncy.cn/} %you can play with color of the template (red is also nice..)
\setgithubaccount{https://github.com/dwdwdw5163} %you can play with color of the template (red is also nice..)
\setthemecolor{red} %you can play with color of the template (red is also nice..)

\begin{document}
	%Set variables
	%You can add sections, texts, explanations just by copying the style below. Replace the dummy texts "\lipsum[1][x-x]\par" with actual texts.
	%Create header
	\headerview
	\vspace{1ex}
	%Sections
	%
	% Summary
	\addblocktext{Summary}{
		\fontsize{9pt}{24pt}对未知事物有着强烈的好奇心,有自我驱动的学习力和找到问题答案的能力。%replace this part with actual text
	}
	%
	%Education
	\section{Education} 
	\datedexperience{\fontsize{10pt}{24pt}武汉理工大学 Wuhan University Of Technology}{2019.9-至今} 
	\explanation{\fontsize{9pt}{24pt}电子科学与技术,本科三年级在读} 
	\explanationdetail{\coloredbullet\ % 
		{\fontsize{9pt}{24pt}担任校电子科技协会副部长,GPA:2.92(79/100),专业排名:43/114}
	}
	\explanationdetail{\coloredbullet\ % 
		{\fontsize{9pt}{24pt}\textbf{主攻方向:FPGA边缘加速IP设计、FPGA神经网络加速、嵌入式应用、Hbirdv2-E203 RISC-V}} 
	}
	\explanationdetail{\coloredbullet\ % 
		{\fontsize{9pt}{24pt}\textbf{期望研究方向:图像处理IP设计、可重构计算、并行加速、嵌入式}} 
	}
	%
		\section{Achievements}
	\newcommand{\extraone}{%
		{\fontsize{9pt}{24pt}基于多模态深度学习的皮肤检测系统[J]. 数码设计(下),2020,9(10):277.} \hfill \datetext{2020}%replace this part with actual text
	}
	%
	\newcommand{\extratwo}{%
		{\fontsize{9pt}{24pt}全国大学生工程训练与综合能力竞赛 \qquad\qquad\qquad\qquad\qquad \textbf{全国二等奖}} \hfill \datetext{2021}%replace this part with actual text
	}
	%
	\newcommand{\extrathree}{%
		{\fontsize{9pt}{24pt}"TI杯"全国大学生电子设计竞赛 \qquad\qquad\qquad\qquad\qquad\qquad\ \, \textbf{全国一等奖}} \hfill \datetext{2021}%replace this part with actual text
	}
	%
	\newcommand{\extrafour}{%
		{\fontsize{9pt}{24pt}第五届全国大学生集成电路创新创业大赛 \qquad\qquad\qquad\quad \ \ \textbf{华中赛区三等奖}} \hfill \datetext{2021}%replace this part with actual text
	}
	%
	\newcommand{\listofextras}{\extraone, \extratwo, \extrathree, \extrafour}
	%
	\createbullets{\listofextras}
	% Experience
	\section{Experience}
	%
	\datedexperience{\fontsize{10pt}{24pt}参与论文多模态深度学习的皮肤检测系统}{2020} 
	\explanationdetail{\coloredbullet\ %
		\fontsize{9pt}{24pt}使用卷积神经网络架构VGGNet-19,了解了有关神经网络的基本知识,为后续的模式识别与机器学习打下基础。
	}
	%
	\datedexperience{\fontsize{10pt}{24pt}全国大学生FPGA创新设计大赛}{2020} 
	\explanationdetail{\coloredbullet\ %
		\fontsize{9pt}{24pt}开始接触FPGA,兴趣点开始从嵌入式单片机慢慢转移到FPGA上,因为现代FPGA的可重构性越来越强, zynq等系列FPGA的出现加快了开发的速度。同时随着深度学习的发展,FPGA的优势也越显突出。为我后续的FPGA学习奠定了基础
	}
	%
	\datedexperience{\fontsize{10pt}{24pt}中国大学生工程实践与创新能力大赛(智能无人机运送)}{2021} 
	\explanationdetail{\coloredbullet\ %
		\fontsize{9pt}{24pt}使用Jetson nano和YOLO3进行目标检测,熟悉了神经网络的部署以及Linux的使用。之后深入学习神经网络并且主要使用Linux开发环境。
	}
	%
	\datedexperience{\fontsize{10pt}{24pt}全国大学生集成电路创新创业大赛(边缘加速IP设计)}{2021} 
	\explanationdetail{\coloredbullet\ %
		\fontsize{9pt}{24pt}制作基于边缘检测的PCB AOI系统。掌握了Vivado和Vitis HLS的使用。熟悉Zynq PS和PL的协同运作。
	}
	%
	\datedexperience{\fontsize{10pt}{24pt}全国大学生电子设计竞赛(用电器分析识别装置)}{2021} 
	\explanationdetail{\coloredbullet\ %
		\fontsize{9pt}{24pt}利用电表芯片HT7038+STM32F4完成电网环境的检测分析,并通过数据处理分析用电器种类和数量,具有学习能力
	}
	%
	\datedexperience{\fontsize{10pt}{24pt}准备软件杯、集创赛}{2022-至今} 
	\explanationdetail{\coloredbullet\ %
		\fontsize{9pt}{24pt}使用mtcnn+facenet进行人脸检测和识别,并在国产嵌入式操作系统SylixOS上通过ncnn+qt+opencv进行部署;设计基于马尔科夫随机场的边缘先验建模的图像上采样(SR-GPP),并依此设计FPGA图片上采样加速IP
	}
	% Skills
	\section{Skills}
	%
	\newcommand{\skillone}{\createskill{\fontsize{9pt}{24pt}编程语言}{\textbf{\emph{Experienced:}} \ \  C/C++ \cpshalf Verilog HDL \cpshalf Python \ \ \textbf{\emph{Familiar:}} \ \  Pytorch \cpshalf Verdi \cpshalf System Verilog}}
	%
	\newcommand{\skilltwo}{\createskill{\fontsize{9pt}{24pt}开发平台}{Windows/Linux \cpshalf STM32 \cpshalf ESP系列 \cpshalf FPGA/ZYNQ \cpshalf Raspberry Pi/Lichee Pi \cpshalf K210 \cpshalf Jetson Nano}}
	%
	\newcommand{\skillthree}{\createskill{\fontsize{9pt}{24pt}语言能力}{\fontsize{9pt}{24pt}英语具有良好的读写能力、阅读过大量英文论文和技术手册、通过CET-6、目前在学习GRE和托福}}
	%
	\createskills{\skillone, \skilltwo, \skillthree}
	% Experience

	%
	%Footnote
	\createfootnote
\end{document}